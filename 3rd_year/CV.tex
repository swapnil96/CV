\documentclass[11pt,a4paper]{moderncv}
\moderncvtheme[blue]{classic} 
\usepackage[utf8]{inputenc}  %Windows 
%\usepackage[scale=0.975]{geometry}
\usepackage[top=0.5cm, bottom=0.5cm, left=0.5cm, right=0.5cm]{geometry}
\usepackage{graphicx}
\firstname{Swapnil}
\familyname{Das}
\title{3rd year B.Tech student}         
%\title{Sophomore B.Tech student}         
\address{SD-18, Kumaon House}{IIT Delhi, New Delhi, 110016}
\mobile{+91 7065158520}
%\phone{(000) 111 1112}
%\fax{(000) 111 1113}
\homepage{http://www.cse.iitd.ac.in/\textasciitilde cs1150263/}
%\email{cs1150263@iitd.ac.in} 
\email{dasswapnil96@gmail.com} 
%\fax{Votre Fax}                      
%\extrainfo{}
%\makeatletter
%\renewcommand*{\bibliographyitemlabel}{\@biblabel{\arabic{enumiv}}}
%\makeatother
%\usepackage{multibib}
%\newcites{book,misc}{{Books},{Others}}                        
\begin{document}
\maketitle
\section{Academic Details}
\cventry{2015--2019}{Bachelor Of Technology In Computer Science Engineering}{\newline Indian Institute of Technology Delhi}{}{\textit{CGPA -- 8.474}}{}
\cventry{2015}{CBSE AISSCE}{Hope Hall Foundation School, Delhi}{}{\textit{Percentage -- 95.8\%}}{}
\cventry{2013}{CBSE AISSE}{Jawahar Navodaya Vidyalaya, Assam}{}{\textit{CGPA -- 10}}{} 

\section{Relevant Courses}
\cvline{Computer Science}{Data Structures and Algorithms, Discrete Mathematics, Digital Logic and System Design, Introduction to Computer Science, Programming Languages, Computer Architecture, Design Practices.}
\cvline{Mathematics}{Probability and Stochastic Processes, Calculus, Linear
Algebra and Differential Equations.}
\cvline{Electrical}{Introduction To Electrical Engineering, Signals \& Systems.}
\cvline{Online}{Introduction to Computer Networking(Stanford), Parallel Programming(EPFL), CS50(Harvard).}
%\cvline{}{\scriptsize *Courses currently ongoing.}

%\section{Projects}
\section{Projects}
\cventry{May - present}{Improvement of Rate Adaptation in 802.11 ac}{ Under Prof. Ben Leong}{ SOC, NUS, Singapore.}{}
{{\begin{itemize}
\item{ Implementing Dynamic Channel Width changes per frame basis to see the effect in overall throughput.}
\item{Implemented in Qualcomm, Intel, Marvell chips with atheros 10k, QCA9880, mwlwifi drivers respectively.}
\end{itemize}}}

\cventry{May - present}{Power optimisation in heterogenous clusters}{ Under Prof. Ben Leong}{ SOC, NUS, Singapore.}{}
{{\begin{itemize}
\item{Reduce power consumption in nodes by distributing tasks to nodes with low requirements of power.}
\item{Experimenting with 5 Raspberry Pis’ and 2 servers as a heterogenous cluster using Kubernetes in Docker.}
\end{itemize}}}

\cventry{December,2016 - May,2017}{Find sections from a PDF}{Under Prof. Dr. Maya Ramanath}{CSE, IIT Delhi}{}
{{\begin{itemize}
\item{Developed a model in Python using NLTK module to find different sections from research papers.}
\item{Regex matching and Naive Bayes methods were used for finding patterns and Parts of Speech tagging respectively.}
\end{itemize}}}

\cventry{September - October, 2016}{Search Engine by Inverted Search Index}{Under Prof. Dr. Amitabha Bagchi}{CSE, IIT Delhi}{}
{{\begin{itemize}
\item{Developed a search engine model in Java using Inverted Search Index.}
\item{Summation of 1/(square of the indexes) where words appeared, was used to calculate relevance.}
\end{itemize}}}

\cventry{August - September, 2016}{Mobile Phone Tracking}{Under Prof. Dr. Amitabha Bagchi}{CSE, IIT Delhi}{}{{\begin{itemize}\item{Developed a multithreaded model in Java to replicate real life mobile tracking  in a area.}
\item{Analysed the Two-way-Radio system of connection.}\end{itemize}}}

\section{Independent Projects} 
\cventry{October - November, 2016}{Proxy/Server-client model}{Python}{Sockets, Threading}{https://github.com/swapnil96/Proxy-Server}{
{\begin{itemize}
\item{Created a server-client model where each server acts as a proxy or delivers the files(any type) to the clients.}
\item{Data is streamed continuously from server through the proxies.}\end{itemize}}}
\cventry{May - July, 2016}{Peer-to-peer networking}{Python}{Sockets}{ Link - https://github.com/swapnil96/P2P-chat}{
{\begin{itemize}
\item{Created a P2P service where multiple clients log into a central server and eventually connect to other clients.}
\item{Server authenticates and manages multiple chat sessions. }
\item{Analysed crowding and high bandwidth consumption in network by connecting 8 systems simultaneously.}\end{itemize}}}
%\cventry{March - July, 2016}{Personal Website}{HTML5}{CSS3, Javascript}{}{{\begin{itemize}
%\item{Created my own website using Bootstrap framework and jquery.}\end{itemize}}}
%\cventry{Jan - Feb, 2016}{GUI of Depth First Search}{Python}{Tkinter.Link - https://github.com/swapnil96/Miscellaneous/blob/master/maze.py}{}{Implemented Python Tkinter module to make a maze where a solution is found from start to end using DFS.}

\section{Technical Skills}
\cvline{Languages}{Python, Java, C++, C, Ocaml, Lex, Yacc, VHDL, HTML, CSS3, MySql, Javascript}
\cvline{Frameworks}{Bootstrap, Django, JQuery, Socket, TKinter, Web2Py, NLTK}
\cvline{Softwares}{GIT, Xilinx ISE suite, Autodesk Inventor, Android Studio, Visual Studio, \LaTeX, MS office}


\section{Scholastic Achievements}
\cvline{2015}{Secured \textbf{All India Rank 2666}(category-\textbf{28}) in JEE-Advance, 2015 among \textbf{0.15 Million} candidates.}
\cvline{2015}{Secured \textbf{All India Rank 1728}(category-\textbf{17}) in JEE-Main 2015 among \textbf{1.4 million} candidates.}
\cvline{2015}{Awarded Scholarship of Merit by CBSE for being in \textbf{top 25 among 1 Million} in AISSCE.}
%\cvline{2014}{Attended Science Camp at Kolkata(India) organized by \textbf{IISc and IISER}.}
\cvline{2014}{Selected as \textbf{(KVPY)} scholar under 'Kishore Vaigyanik Protsahan Yojana' given to top 1\% conducted by \textbf{IISc} Bangalore and \textbf{Govt. of India.}}
\cvline{2013}{Awarded Scholarship of Merit by CBSE for being in \textbf{top 27 among 1.3 Million} in AISSE.}
%\cvline{2013}{Qualified to appear in \textbf{Indian National Mathematics Olympiad(INMO)} conducted by \textbf{Homi Bhabha Centre for Science Education(HBCSE)}.}
% \cvline{2013}{Selected as Dakshana scholar for qualifying in GDST conducted by \textbf{HRD ministry Govt. of India}.}
\cvline{2012}{Qualified \textbf{Regional Mathematics Olympiad(RMO)} and appeared in \textbf{Indian National Mathematics Olympiad(INMO)} conducted by \textbf{Homi Bhabha Centre for Science Education(HBCSE)}.}

\section{Extra Curricular Achievements}
\subsection{Programming Contest}
\cvline{Jan 2016}{\textbf{Microsoft Code.Fun.Do Hakathon} - Runners up. Created a game in Visual Studio using C\#}
\cvline{July 2016}{\textbf{Microsoft Finalist Forum} - Participated online and completed 1st round.}
%\cvitem{Codeforces}{
%Handle: \textbf{dasswapnil96}, {}
%Max Rating: 1410}
%\cvitem{Codechef}{
%Handle: \textbf{dasswapnil96}, {}
%Rank: 3109 - Global, 2335 - India}
\subsection{Others}
\cvline{2013}{3rd place in Intra District debate competition.}
\cvline{2012}{Regional Badminton player from JNV under Shillong Region.}
\cvline{2002 - 2009}{Completed a 7 year course in Fine – Arts.}

\section{Position of Responsibility}
\cvline{2017-18}{Students Affairs Council representative from Kumaon hostel.}
\cvline{2016-17}{Website executive of eDC team, IIT Delhi.}
\cvline{2016-17}{Activity Head in TRYST(Annual technical festival) 2017 of IIT Delhi.}

%\section{References}
%\cventry{}{Rupam Barman}{IIT Guwahati, Assam, rupam@iitg.ernet.in.}{}
%{\newline Associate Professor in Mathmetics department IIT Guwahati.}{}
\end{document}